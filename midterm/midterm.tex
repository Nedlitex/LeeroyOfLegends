\documentclass{article} % For LaTeX2e
\usepackage{midterm,times}
\usepackage{hyperref}
\usepackage{url}
%\documentstyle[nips13submit_09,times,art10]{article} % For LaTeX 2.09


\title{A Machine Learning Approach For E-sports Analysis}


\author{
Xiao Bo Zhao\\
Carnegie Mellon University\\
Pittsburgh, PA 15213 \\
\texttt{xiaoboz@andrew.cmu.edu} \\
\And
Shen Wang\\
Carnegie Mellon University\\
Pittsburgh, PA 15213 \\
\texttt{shenw@andrew.cmu.edu} \\
}

% The \author macro works with any number of authors. There are two commands
% used to separate the names and addresses of multiple authors: \And and \AND.
%
% Using \And between authors leaves it to \LaTeX{} to determine where to break
% the lines. Using \AND forces a linebreak at that point. So, if \LaTeX{}
% puts 3 of 4 authors names on the first line, and the last on the second
% line, try using \AND instead of \And before the third author name.

\newcommand{\fix}{\marginpar{FIX}}
\newcommand{\new}{\marginpar{NEW}}

\nipsfinalcopy % Uncomment for camera-ready version

\begin{document}


\maketitle

\begin{abstract}
In recent years, e-sport has gained its popularity, but the study of it has been lagged behind. Therefore, in this project, we look deep into the nature of e-sport and create models that can best predict its outcome. We look at the well studied models targeting the traditional sports and evaluate our approach against the existing models.
\end{abstract}

%---------------------------------------------------------
\section{Introduction}
In the last couple of years, e-sport saw a significant growth, both in terms of players and spectators \cite{Witkowski:2013:ERC:2513002.2513008}. Unlike traditional sports, online matches are formed quickly. Players can find and play against opponents within minutes. As a result, e-sport competitive play occurs very frequently. This fluidity also causes e-sports’ structure to differ significantly from traditional sports. To allow for fast paced match making in games like League of Legends, both teammates and opponents are chosen randomly from a large pool of players of similar ranking. Players on the same team most likely do not have prior collaboration history, while players facing against each other most likely have not studied each other’s strategies. Because of this added complexity, many well-studied models that predict traditional sports’ outcomes might no longer apply to e-sports. Therefore, in this study, we intend to develop a classifier adapted for e-sports’ structure, and in particular, for League of Legend’s competitive gaming model.

%---------------------------------------------------------

\section{Related work}
Given the similarities between e-sports and traditional sports, we believe that it is beneficial to look at some well studied models targeting traditional sports and use them as our baseline. There are many related works of this field~\cite{Joseph2006544}\cite{Min2008551}, and each of them provide unique angles to the problem. One particularly interesting model aimed to predict the outcome of football matches proposed by {A. Joseph et al.} is constructed based on Bayesian networks~\cite{Joseph2006544}. In this model, a Bayesian network is constructed based on the presence of key players, their positions, home-field benefit and an estimation of the quality of the opposing team. As far as we are concerned, this model has several important points. First of all, due to the nature of football game organizations, there is not much data to consider. That said, a particular football player only has a limited career time and there are only limited amount of football games in a season. As a result, for each player, their performance data are very limited. The insufficiency of data suggests that it is inappropriate to build models based on each player's performance in the games. To solve this problem, the proposed model only consider the presence of some key players as well as whether their initial playing positions. The observation is that a player, with his professional skills, will contribute to the team in the game - as long as he plays in that game. Another important design in the proposed model is the asymmetry between two teams in a game. As described, the players in one team will be checked specifically by the model while the other team is only represented as a general quality score assigned by an "expert". This design makes it possible for the "expert" to drive the direction of prediction. The 
%---------------------------------------------------------

\section{Methods}
vvv
%---------------------------------------------------------

\section{Experiments}

A table:

\begin{table}[t]
\caption{Sample table title}
\label{sample-table}
\begin{center}
\begin{tabular}{ll}
\multicolumn{1}{c}{\bf PART}  &\multicolumn{1}{c}{\bf DESCRIPTION}
\\ \hline \\
Dendrite         &Input terminal \\
Axon             &Output terminal \\
Soma             &Cell body (contains cell nucleus) \\
\end{tabular}
\end{center}
\end{table}
%---------------------------------------------------------

\section{Conclusion}

vvv
%---------------------------------------------------------

\section{References}

\bibliographystyle{midterm}
\bibliography{midterm}

\end{document}